\documentclass[hyperref, UTF-8]{ctexart}
\usepackage{lastpage}
\usepackage{fancyhdr}
\pagestyle{fancy}
\renewcommand{\headrulewidth}{0.4pt} 
\renewcommand{\footrulewidth}{0.4pt}
\fancyhead[LE,RO]{same as title}
\fancyhead[LE,LO]{\thepage}
\fancyfoot[CE,CO]{\leftmark}
\fancyfoot[LE,RO]{\thepage\ of \pageref{LastPage}}
\usepackage{makeidx}
\usepackage[colorlinks,
            linkcolor=red,
            anchorcolor=blue,
            citecolor=green,
            CJKbookmarks
            ]{hyperref}
%\usepackage[center]{titlesec} 
\author{kay}
\title{Machine Learning Activations}
\makeindex
\begin{document}
\maketitle
\tableofcontents
\section{Introduction}
当神经网络的激活函数均为线性函数的时候会有什么样的结果;那么最终输出与最开始的输
入必然也是线性的关系,这样的话网络中的隐层就没有体现其隐层的价值。这样以来,隐层
的激活函数必然需要非线性的函数来进行表示。激活函数是用来加入非线性因素的

\href{https://www.jiqizhixin.com/articles/2017-10-10-3}{26种神经网络激活函数可视化}
\href{https://dashee87.github.io/data%20science/deep%20learning/visualising-activation-functions-in-neural-networks/}{Visualising Activation Functions in Neural Networks}
\section{softmax}
Softmax regression (or multinomial logistic regression) is a generalization of
logistic regression to the case where we want to handle multiple classes.
Softmax regression allows us to handle $y^{(i)}∈{1,…,K}$ where K is the number
of classes.

$\mathcal{K}$-dimensional vector $\mathrm{z}$ of arbitrary real values to a
$\mathcal{K}$-dimensional vector $\sigma(\mathrm{z})$ of real values in the
range [0, 1] that add up to 1.
This function is given by:
\begin{displaymath}
  \sigma(z)_j = \frac{e^{z_{j}}{\sum_{i=1}^{K} e^{z_k}} ,  for j = 1, ..., K
\end{displaymath}


Hypothesis took the form ~\cite{softmax1}:
\begin{displaymath}
%   Todo: fix this
  \mathit{h}_\theta(x) = ...
\end{displaymath}
and the model parameters $\theta$ were trained to minimize the cost function:
\begin{displaymath}
 % Todo: fix this function
  \rm{J}(\theta) = ...
\end{displaymath}

\begin{thebibliography}{99}
\bibitem{softmax1}
  \href{http://ufldl.stanford.edu/tutorial/supervised/SoftmaxRegression/}{softmax
  regression}
  
\end{thebibliography}

\end{document}
