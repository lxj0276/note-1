\documentclass[hyperref, UTF-8]{ctexart}
\usepackage{amssymb}
\usepackage{lastpage}
\usepackage{fancyhdr}
\pagestyle{fancy}
\renewcommand{\headrulewidth}{0.4pt} 
\renewcommand{\footrulewidth}{0.4pt}
\fancyhead[LE,RO]{Machine Learning Loss Function}
\fancyhead[LE,LO]{\thepage}
\fancyfoot[CE,CO]{\leftmark}
\fancyfoot[LE,RO]{\thepage\ of \pageref{LastPage}}
\usepackage{makeidx}
\usepackage[colorlinks,
            linkcolor=red,
            anchorcolor=blue,
            citecolor=green,
            CJKbookmarks
            ]{hyperref}
%\usepackage[center]{titlesec} 
\author{kay}
\title{Machine Learning Optimizition}
\makeindex
\begin{document}
\maketitle
\tableofcontents

\section{Introduction}

损失函数是经验风险函数的核心部分,也是结构风险函数重要组成部分。模型的结构风险函数包括了经验风险项和正则项,通常可以表示成如下式子:
\begin{displaymath}
  \theta{^\star} = arg \min{_\theta} \frac{1}{N}\sum{i=1}^N \mathit{L} (y_i,
  f(x_i; \theta)) + \lambda \Theta(\theta)
\end{displaymath}
其中,前面的均值函数表示的是经验风险函数,L代表的是损失函数,后面的$\Theta$是正
则化项(regularizer)或者叫惩罚项(penalty term)

\subsection{Loss Functions Comparison}
Given a prediction (p) and a label (y), a loss function loss function
$\ell(p,y)$ measures the discrepancy between the algorithm's prediction and the
desired output.   \\
\begin{tabular}{|c|c|c|c|}
  \hline
  Loss & Function & Minimizer & Example usage \\ \hline
  Squared & $\frac{1}{2}(p-y)^2$ & Expectation(mean) & Regression Expected return on stock \\  \hline
\end{tabular}

\section{Gold Standard Loss}
Gold Standard又称0-1误差,其结果又称为犯错与不犯错,用途比较广(比如PLA模型),其损失函数也是相当的简单:
\begin{displaymath}
  y = \left \{
      \begin{array}{ll}
        0 & if ~ m \geqslant 0  \\
        1 &  if ~ m \leqslant 0 
      \end{array}  \right.
\end{displaymath}

\section{Hinge Loss}

\section{Log Loss}
Cross entropy Loss is a type of Log Loss.

\section{Squared Loss}

\section{Exponential Loss}



\end{document}
