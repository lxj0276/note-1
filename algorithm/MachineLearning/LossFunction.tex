\documentclass[hyperref, UTF-8]{ctexart}
\usepackage{amssymb}
\usepackage{lastpage}
\usepackage{fancyhdr}
\pagestyle{fancy}
\renewcommand{\headrulewidth}{0.4pt} 
\renewcommand{\footrulewidth}{0.4pt}
\fancyhead[LE,RO]{Machine Learning Loss Function}
\fancyhead[LE,LO]{\thepage}
\fancyfoot[CE,CO]{\leftmark}
\fancyfoot[LE,RO]{\thepage\ of \pageref{LastPage}}
\usepackage{makeidx}
\usepackage[colorlinks,
            linkcolor=red,
            anchorcolor=blue,
            citecolor=green,
            CJKbookmarks
            ]{hyperref}
%\usepackage[center]{titlesec} 
\author{kay}
\title{Machine Learning Optimizition}
\makeindex
\begin{document}
\maketitle
\tableofcontents

\section{Introduction}
In mathematical optimization, a loss function or cost function is a function
that maps an event or values of one or more variables onto a real number
intuitively representing some "cost" associated with the event. in statistics,
typically a loss function is used for parameter estimation. In classification,
it is the penalty for an incorrect classification of an example. In actuarial
science, it is used in an insurance context to model benefits paid over
premiums. In optimal control the loss is the penalty for failing to achieve a
desired value. In financial risk management the function is mapped to a monetary
loss.\cite{lfwiki} 
\subsection{Definition}
Given a set A of possible actions, a decision rule is a function $\delta : \mathcal{X} \rightarrow \mathbf{A}$.

A loss function is a real lower-bounded function L on $\theta \times mathrm{A}$
for some $\theta \in \Theta$. The value $L(\theta, \delta(X))$ is the cost of
action $\delta{(\mathit{X})}$ under parameter $\theta$. 

损失函数是经验风险函数的核心部分,也是结构风险函数重要组成部分。模型的结构风险函数包括了经验风险项和正则项,通常可以表示成如下式子:
\begin{displaymath}
  \theta{^\star} = arg \min{_\theta} \frac{1}{N}\sum{i=1}^N \mathit{L} (y_i,
  f(x_i; \theta)) + \lambda \Theta(\theta)
\end{displaymath}
其中,前面的均值函数表示的是经验风险函数,L代表的是损失函数,后面的$\Theta$是正
则化项(regularizer)或者叫惩罚项(penalty term)

\subsection{Loss Functions Comparison}
Given a prediction (p) and a label (y), a loss function loss function
$\ell(p,y)$ measures the discrepancy between the algorithm's prediction and the
desired output.   \\
\begin{tabular}{|c|c|c|c|}
  \hline
  Loss & Function & Minimizer & Example usage \\ \hline
  Squared & $\frac{1}{2}(p-y)^2$ & Expectation(mean) & Regression Expected return on stock \\  \hline
\end{tabular}

\section{Gold Standard Loss}
Gold Standard又称0-1误差,其结果又称为犯错与不犯错,用途比较广(比如PLA模型),其损失函数也是相当的简单:
\begin{displaymath}
  y = \left \{
  \mathbf{L}(y, f(x)) = \left \{
      \begin{array}{ll}
        0 & if ~ m \geqslant 0  \\
        1 &  if ~ m \leqslant 0 
        1, & if ~ y\neq f(x) \\
        0, &  if ~ y = f(x) 
      \end{array}  \right.
\end{displaymath}

0-1损失是一个非凸的函数,在求解的过程中,存在很多的不足,通常在实际的使用中将0-1损失函数作为一个标准,选择0-1损失函数的代理函数作为损失函数。
$\mathbf{L}(\hat{y}, y) = \mathit{I}(\hat{y} \neq y)$, where $\mathit{I}$ is the
\href{https://en.wikipedia.org/wiki/Indicator_function}{ indicator notation }\cite{loss}.

\section{Hinge Loss}
\begin{eqnarray*}
  y_i X_i \cdot \mathbf{W} \geq 1 - \epsilon{_i}, \quad \epsilon_i \geq 0 \quad
  \forall{A}  \\
  => \quad \epsilon_i \geq 1 - y_iX_i\cdot \mathbf{W}, \quad \epsilon_i \geq 0
  \quad \forall{A}  \\
  => \quad \epsilon_i \geq max{0, 1-y_iX_i\cdot\mathbf{W}}, \quad \forall{A}
\end{eqnarray*}
Hinge损失函数的标准形式: 
\begin{displaymath}
  L(y)= max(0, 1-y\hat{y}),\quad y=\pm{1}
\end{displaymath}
SVM损失函数:
\begin{displaymath}
  \min_{\omega,b}\frac{1}{C}\left( \frac{1}{2} \parallel\omega\parallel^2 + C
  \sum_{i}^N \epsilon_i \right)
\end{displaymath}
\subsection{Huber Loss}


\section{Log Loss}
Cross entropy Loss is a type of Log Loss.
For discrete $\mathit{p}$ and $\mathit{q}$ this means~\cite{logloss}:
\begin{displaymath}
  H(p, q) = - \sum_{x}p(x)\log{ q(x) }
\end{displaymath}
For situation for continuous distributions:
\begin{displaymath}
  -\int_XP(x)\log{Q(x)} dr(x) = E_p[-logQ]
\end{displaymath}

\subsection{Logistic Function}
The logistic regression model thus predicts an output $y \in \{0, 1\}$, given an
input vector $\mathbf {x}$ ~\cite{logloss}. The probability is modeled using the logistic function $g(z)=1/(1+e^{-z})$. Namely, the probability of finding the output  $y=1$ is given by
\begin{displaymath}
  q_{y=1}\ = \ \hat{y}\ \equiv \ g(\mathbf{w} \cdot \mathbf{x})
\end{displaymath}
where the vector of weights $\mathbf {w}$  is optimized through some appropriate
algorithm such as \emph{gradient descent}. Similarly, the complementary probability of
finding the output $y=0$ is simply given by $q_{{y=0}}\ =\ 1-{\hat  {y}}$.

We can use cross entropy to get a measure for similarity between $p$ and $q$:
\begin{displaymath}
H(p,q)\ =\ -\sum _{i}p_{i}\log q_{i}\ =\ -y\log {\hat {y}}-(1-y)\log(1-{\hat {y}})
\end{displaymath}

Suppose we have $N$ samples with each sample labeled by $n=1,\dots ,N$. The loss function is then given by:
\begin{displaymath}
 L(\mathbf {w} )\ =\ {\frac {1}{N}}\sum _{n=1}^{N}H(p_{n},q_{n})\ =\ -{\frac
   {1}{N}}\sum _{n=1}^{N}\ {\bigg [}y_{n}\log {\hat
   {y}}_{n}+(1-y_{n})\log(1-{\hat {y}}_{n}){\bigg ]}
\end{displaymath}
where ${\hat  {y}}_{n}\equiv g({\mathbf  {w}}\cdot {\mathbf  {x}}_{n})$, with $g(z)$ the logistic function as before.

The logistic loss is sometimes called cross-entropy loss. It is also known as log loss (In this case, the binary label is often denoted by {-1,+1}).
\section{Squared Error}
The MSE is a measure of the quality of an estimator—it is always non-negative, and values closer to zero are better.~\cite{mseloss}

The MSE is the second moment (about the origin) of the error, and thus
incorporates both the variance of the estimator and its bias.

If $\hat{Y}$ is a vector of $n$ predictions, and $Y$ is the vector of observed values corresponding to the inputs to the function which generated the predictions, then the MSE of the predictor can be estimated by
\begin{displaymath}
 {MSE}={\frac  {1}{n}}\sum_{{i=1}}^{n}({\hat  {Y_{i}}}-Y_{i})^{2}
\end{displaymath}
I.e., the MSE is the mean \begin{math}\left({\frac {1}{n}}\sum_{i=1}^{n}\right)\end{math} of the square of the errors ($({\hat  {Y_{i}}}-Y_{i})^{2}$).

The MSE can be written as the sum of the variance of the estimator and the
squared bias of the estimator:
\begin{displaymath}
{MSE} ({\hat {\theta }})= {Var} ({\hat {\theta }})+ {Bias} ({\hat {\theta }},\theta )^{2}.  
\end{displaymath}
\section{Exponential Loss}

\section{Absolute Loss}

\section{Perceptron Loss}
感知机损失是对0-1损失设了一个阈值
\begin{displaymath}
  \ell{( y_i, \hat{y_i} )} = \left\{
      \begin{array}{ll}
        1,&|y_i - \hat{y_i}| > t \\
        2,& |{y_i = \hat{y_i}}| \leq t
      \end{array} 
\right.
\end{displaymath}
其中t是一个超参数阈值,如在PLA(Perceptron Learning Algorithm,感知机算法).

\begin{thebibliography}{99}
\bibitem{loss}
  \href{https://en.wikipedia.org/wiki/Loss_function#0-1_loss_function}{0-1 loss
    function wikipedia}
\bibitem{logloss}
  \href{https://en.wikipedia.org/wiki/Cross_entropy#Cross-entropy_error_function_and_logistic_regression}{Cross
    entropy wiki}
\bibitem{mseloss} \href{https://en.wikipedia.org/wiki/Mean_squared_error}{Mean squared error}
\end{thebibliography}

\end{document}
