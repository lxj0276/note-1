\documentclass[hyperref, UTF-8]{ctexart}
\usepackage{amssymb}
\usepackage{lastpage}
\usepackage{fancyhdr}
\pagestyle{fancy}
\renewcommand{\headrulewidth}{0.4pt} 
\renewcommand{\footrulewidth}{0.4pt}
\fancyhead[LE,RO]{Machine Learning Optimizition}
\fancyhead[LE,LO]{\thepage}
\fancyfoot[CE,CO]{\leftmark}
\fancyfoot[LE,RO]{\thepage\ of \pageref{LastPage}}
\usepackage{makeidx}
\usepackage[colorlinks,
            linkcolor=red,
            anchorcolor=blue,
            citecolor=green,
            CJKbookmarks
            ]{hyperref}
%\usepackage[center]{titlesec} 
\author{kay}
\title{Machine Learning Optimizition}
\makeindex
\begin{document}
\maketitle
\tableofcontents
\section{Optimization Methods for Large-Scale Machine Learning}
\href{https://arxiv.org/abs/1606.04838}{Optimization Methods for Large-Scale Machine Learning}
\subsection{Introduction}
Overall, this paper attempts to provide answers for the following questions.
\begin{enumerate}
\item{How do optimization problems arise in machine learning applications and
    what makes them challenging?}
\item{What have been the most successful optimization methods for large-scale
    machine learning and why?}
\item{what recent advances have been made in the design of algorithms and what
    are open questions in this research area?}
\end{enumerate}
For answer question one, use text classification(involves \emph{convex}
optimization problems); for answer question two, use speech or image
recognition(involves \emph{highly nonlinear and nonconver} problems)

In this paper, disguss: (i) noise reduction methods that attempt to borrow from
the strengths of batch methods, such as their fast convergence rates and ability
to exploit parallelism; (ii) methods that incorporate approximate second-order
derivative information with the goal of dealing with nonlinearity and
ill-conditioning; and (iii) methods for solving regularized problems designed to
avoid overfitting and allow for the use of high-dimensional models.
 
\subsection{Machine Learning Case Studies}
We focus on two cases that involve very large datasets and for which the number of model parameters to be optimized is also large. 

\subsubsection{Text Classification via Convex Optimization}
Splitting the examples into three disjoint subsets: a \emph{training set}, a \emph{validation set}, and a \emph{testing set}. 
The generalized performance of each of these remaining candidates is then estimated using the validation set, the best performing of which is chosen as the selected function. The testing set is only used to estimate the generalized performance of this selected function.
Convex optimization problem:
\begin{equation}
  \min_{(\omega,\tau)\in \mathbb{R}^d\times \mathbb{R}}\frac{1}{n}\sum_{i=1}^{n}\ell(h(x_i;\omega,\tau),y_i) + \frac{\lambda}{2}\parallel\omega\parallel_2^2
\end{equation}
This problem may be solved multiple times for a given training set with various values of $\lambda$ > 0, with the ultimate solution ($\omega${$\ast$},$\tau${$\ast$}) being the one that yields the best performance on a validation set.

\subsubsection{Perceptual Tasks via Deep Neural Networks}
Canonical fully connected layer performs the computation
\begin{equation}
  x_i^{(j)} = \mathrm{ s }(\mathrm{ W }_j x_i^{(j-1)}+b_j) \in \mathbb{R}^{d_j}
\end{equation}
Choice of a loss function $\ell$ leading to
\begin{equation}
  \min_{\omega\in\mathbb{R}^d} \frac{1}{n} \sum_{i=1}^n \ell(h(x_i;\omega), y_i)
\end{equation}

\section{DML Optimization}
\href{https://zhuanlan.zhihu.com/p/22252270}{深度学习最全优化方法总结比较(SGD,Adagrad,Adadelta,Adam,Adamax,Nadam)}

BGD(batch gradient descent)
\begin{eqnarray*}
  \hat{g} \leftarrow + \frac{1}{n} \nabla_{\theta}\sum_i
  \mathbf{L}(f(x_i;\theta),y_i) \\
  \theta \leftarrow \theta - \epsilon \hat{g}
\end{eqnarray*}
优点: 由于每一步都利用了训练集中的所有数据,因此当损失函数达到最小值以后,能够保证此时计算出的梯度为0,换句话说,就是能够收敛.因此,使用BGD时不需要逐渐减小学习速率ϵk. \\
缺点: 由于每一步都要使用所有数据,因此随着数据集的增大,运行速度会越来越慢.
\subsection{SGD}
SGD(Stochastic Gradient Descent), 一般指Mini-batch Gradient Descent
\begin{eqnarray*}
  g_t = \nabla_{\theta_{t-1}}f(\theta_{t-1})  \\
  \Delta{\theta_t} = -\eta\times g_t
\end{eqnarray*}
$\eta$是学习率, $g_t$是梯度SGD完全依赖于当前batch的梯度, $\eta$可理解为允许当前
batch的梯度多大程度影响参数更新.

缺点:
\begin{itemize}
\item 选择合适的learning rate比较困难 - 对所有的参数更新使用同样的learning rate。对于稀疏数据或者特征,有时我们可能想更新快一些对于不经常出现的特征,对于常出现的特征更新慢一些,这时候SGD就不太能满足要求了
\item SGD容易收敛到局部最优,并且在某些情况下可能被困在鞍点
\item 由于是抽取,因此不可避免的,得到的梯度肯定有误差.因此学习速率需要逐渐减小.否
  则模型无法收敛, 一般选择线性衰减:  \\ $\eta{_k}=(1−\alpha)\eta_0+\alpha\eta_{\tau}$  \\
  $\alpha=\frac{k}{\tau}$
\end{itemize}

\subsection{Momentum}
momentum是模拟物理里动量的概念,积累之前的动量来替代真正的梯度。公式如下:
\begin{eqnarray*}
  m_t = \mu \times m_{t-1} + g_t  \\
  \Delta\theta_t = -\eta \times m_t
\end{eqnarray*}
其中$\mu$是动量因子 

特点:
\begin{itemize}
\item 下降初期时,使用上一次参数更新,下降方向一致,乘上较大的$\mu$能够进行很好的加速
\item 下降中后期时,在局部最小值来回震荡的时候,gradient$\to$0,$\mu$使得更新幅度增大,跳出陷阱
\item 在梯度改变方向的时候,$\mu$能够减少更新 总而言之,momentum项能够在相关方向加速SGD,抑制振荡,从而加快收敛
\end{itemize}

\subsection{Nesterov}
nesterov项在梯度更新时做一个校正,避免前进太快,同时提高灵敏度。 将上一节中的公式展开可得:
\begin{displaymath}
\Delta{\theta_t}=-\eta*\mu*m_{t-1}-\eta*g_t
\end{displaymath}
可以看出,$m_{t-1}$
并没有直接改变当前梯度$g_t$,所以Nesterov的改进就是让之前的动量直接影响当前的动量。即:
\begin{eqnarray*}
g_t=\nabla_{\theta_{t-1}}{f(\theta_{t-1}-\eta*\mu*m_{t-1})}  \\
m_t=\mu*m_{t-1}+g_t  \\
\Delta{\theta_t}=-\eta*m_t 
\end{eqnarray*}
其实,momentum项和nesterov项都是为了使梯度更新更加灵活,对不同情况有针对性。但是,人工设置一些学习率总还是有些生硬,接下来介绍几种自适应学习率的方法

\subsection{Adagrad}
Adagrad其实是对学习率进行了一个约束
\begin{eqnarray*}
n_t=n_{t-1}+g_t^2 \\
\Delta{\theta_t}=-\frac{\eta}{\sqrt{n_t+\epsilon}}*g_t 
\end{eqnarray*}
此处,对$g_t$从1到t进行一个递推形成一个约束项regularizer, \\
$-\frac{1}{\sqrt{\sum_{r=1}^t(g_r)^2+\epsilon}},\epsilon$ 用来保证分母非0 

特点:
\begin{itemize}
\item 前期$g_t$较小的时候, regularizer较大,能够放大梯度
\item 后期$g_t$较大的时候,regularizer较小,能够约束梯度
\item 适合处理稀疏梯度
\end{itemize}

缺点:
\begin{itemize}
\item 由公式可以看出,仍依赖于人工设置一个全局学习率
\item $\eta$设置过大的话,会使regularizer过于敏感,对梯度的调节太大
\item 中后期,分母上梯度平方的累加将会越来越大,使$gradient\to0$,使得训练提前结束
\end{itemize}

\subsection{Adadelta}
Adadelta是对Adagrad的扩展,最初方案依然是对学习率进行自适应约束,但是进行了计算上的简化。 Adagrad会累加之前所有的梯度平方,而Adadelta只累加固定大小的项,并且也不直接存储这些项,仅仅是近似计算对应的平均值。即:
\begin{eqnarray*}
n_t=\nu*n_{t-1}+(1-\nu)*g_t^2  \\
\Delta{\theta_t} = -\frac{\eta}{\sqrt{n_t+\epsilon}}*g_t 
\end{eqnarray*}
在此处Adadelta其实还是依赖于全局学习率的,但是作者做了一定处理,经过近似牛顿迭代法之后:
\begin{eqnarray*}
E|g^2|_t=\rho*E|g^2|_{t-1}+(1-\rho)*g_t^2  \\
\Delta{x_t}=-\frac{\sqrt{\sum_{r=1}^{t-1}\Delta{x_r}}}{\sqrt{E|g^2|_t+\epsilon}}
\end{eqnarray*}
其中,E代表求期望。
此时,可以看出Adadelta已经不用依赖于全局学习率了。 

特点:
\begin{itemize}
\item 训练初中期,加速效果不错,很快
\item 训练后期,反复在局部最小值附近抖动
\end{itemize}

\subsection{RMSprop}
RMSprop可以算作Adadelta的一个特例:

当$\rho$=0.5时,$E|g^2|_t=\rho*E|g^2|_{t-1}+(1-\rho)*g_t^2$就变为了求梯度平方和的平均数。

如果再求根的话,就变成了RMS(均方根):
\begin{displaymath}
RMS|g|_t=\sqrt{E|g^2|_t+\epsilon}
\end{displaymath}

此时,这个RMS就可以作为学习率$\eta$的一个约束:
\begin{displaymath}
\Delta{x_t}=-\frac{\eta}{RMS|g|_t}*g_t
\end{displaymath}

特点:
\begin{itemize}
\item 其实RMSprop依然依赖于全局学习率
\item RMSprop算是Adagrad的一种发展,和Adadelta的变体,效果趋于二者之间
\item 适合处理非平稳目标 - 对于RNN效果很好
\end{itemize}

\subsection{Adam}
Adam(Adaptive Moment Estimation)本质上是带有动量项的RMSprop,它利用梯度的一阶矩估计和二阶矩估计动态调整每个参数的学习率。Adam的优点主要在于经过偏置校正后,每一次迭代学习率都有个确定范围,使得参数比较平稳。公式如下:
\begin{eqnarray*}
m_t=\mu*m_{t-1}+(1-\mu)*g_t \\
n_t=\nu*n_{t-1}+(1-\nu)*g_t^2 \\
\hat{m_t}=\frac{m_t}{1-\mu^t}  \\
\hat{n_t}=\frac{n_t}{1-\nu^t}  \\
\Delta{\theta_t}=-\frac{\hat{m_t}}{\sqrt{\hat{n_t}}+\epsilon}*\eta
\end{eqnarray*}
其中,$m_t$,$n_t$分别是对梯度的一阶矩估计和二阶矩估计,可以看作对期望$E|g_t|$,$E|g_t^2|$的估计;$\hat{m_t}$,$\hat{n_t}$是对$m_t$,$n_t$的校正,这样可以近似为对期望的无偏估计。 可以看出,直接对梯度的矩估计对内存没有额外的要求,而且可以根据梯度进行动态调整,而$-\frac{\hat{m_t}}{\sqrt{\hat{n_t}}+\epsilon}$对学习率形成一个动态约束,而且有明确的范围。

特点:
\begin{itemize}
\item 结合了Adagrad善于处理稀疏梯度和RMSprop善于处理非平稳目标的优点
\item 对内存需求较小
\item 为不同的参数计算不同的自适应学习率
\item 也适用于大多非凸优化 - 适用于大数据集和高维空间
\end{itemize}

\subsection{Adamax}
Adamax是Adam的一种变体,此方法对学习率的上限提供了一个更简单的范围。公式上的变化如下:
\begin{eqnarray*}
n_t=max(\nu*n_{t-1},|g_t|)  \\
\Delta{x}=-\frac{\hat{m_t}}{n_t+\epsilon}*\eta
\end{eqnarray*}
可以看出,Adamax学习率的边界范围更简单

\subsection{Nadam}
Nadam类似于带有Nesterov动量项的Adam。公式如下:
\begin{eqnarray*}
\hat{g_t}=\frac{g_t}{1-\Pi_{i=1}^t\mu_i}  \\
m_t=\mu_t*m_{t-1}+(1-\mu_t)*g_t  \\
\hat{m_t}=\frac{m_t}{1-\Pi_{i=1}^{t+1}\mu_i}  \\
n_t=\nu*n_{t-1}+(1-\nu)*g_t^2  \\
\hat{n_t}=\frac{n_t}{1-\nu^t}\bar{m_t}=(1-\mu_t)*\hat{g_t}+\mu_{t+1}*\hat{m_t}  \\
\Delta{\theta_t}=-\eta*\frac{\bar{m_t}}{\sqrt{\hat{n_t}}+\epsilon}
\end{eqnarray*}
可以看出,Nadam对学习率有了更强的约束,同时对梯度的更新也有更直接的影响。一般而言,在想使用带动量的RMSprop,或者Adam的地方,大多可以使用Nadam取得更好的效果。

\subsection{Experience}
\begin{itemize}
\item 对于稀疏数据,尽量使用学习率可自适应的优化方法,不用手动调节,而且最好采用默认值
\item SGD通常训练时间更长,但是在好的初始化和学习率调度方案的情况下,结果更可靠
\item 如果在意更快的收敛,并且需要训练较深较复杂的网络时,推荐使用学习率自适应的优化方法。
\item Adadelta,RMSprop,Adam是比较相近的算法,在相似的情况下表现差不多。
\item 在想使用带动量的RMSprop,或者Adam的地方,大多可以使用Nadam取得更好的效果
\end{itemize}
\section{Machine Learning Optimization}



\begin{thebibliography}{99}
\bibitem{lfwiki} \href{https://en.wikipedia.org/wiki/Loss_function}{Loss function wiki}
\end{thebibliography}
\end{document}
