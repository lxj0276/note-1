% Created 2017-04-07 Fri 13:57
% Intended LaTeX compiler: pdflatex
\documentclass[11pt]{article}
\usepackage[utf8]{inputenc}
\usepackage[T1]{fontenc}
\usepackage{graphicx}
\usepackage{grffile}
\usepackage{longtable}
\usepackage{wrapfig}
\usepackage{rotating}
\usepackage[normalem]{ulem}
\usepackage{amsmath}
\usepackage{textcomp}
\usepackage{amssymb}
\usepackage{capt-of}
\usepackage{hyperref}
\author{kay}
\date{\today}
\title{}
\hypersetup{
 pdfauthor={kay},
 pdftitle={},
 pdfkeywords={},
 pdfsubject={},
 pdfcreator={Emacs 25.1.1 (Org mode 9.0.5)}, 
 pdflang={English}}
\begin{document}

\tableofcontents

优先级用 A, B, C, D 代替, 从 A 往后优先级依次降低.
\section{hive on spark(耗时的离线批量处理) :A}
\label{sec:orgda377f0}
\begin{enumerate}
\item 将执行时间较长的 hive 运行在 spark 上,调节执行参数.
\item 用 spark sql 改写一些 hive 语句, 查看效果.
\end{enumerate}
\section{spark streaming :A}
\label{sec:orgfa1c121}
\begin{enumerate}
\item spark streaming 做实时需求.
\item spark streaming 做实时算法训练模型. :C
\end{enumerate}
\section{spark 数据处理, 然后写入到 es, hbase 等,做 hdfs, hive, es, hbase 等的数据对接工具 :B}
\label{sec:orgb852ac9}
\section{spark 做一些特殊的数据清洗(用 hive 比较难弄的复杂的数据数据清洗) :B}
\label{sec:org863264d}
当处理日志的半结构化或者非结构化数据时, 对其进行清洗和转换操作时,需要结合 SQL 查询以及复杂的过程式逻辑处理,这部分工作虽然可以由 Hive SQL 结合 Python 或 shell 脚本来完成。这种方式存在效率问题,当数据量比较大的时候,流程的运行时间较长,这些 ETL 流程通常处于比较上游的位置,会直接影响到一系列下游的完成时间以及各种重要数据报表的生成。
如果用 spark 配合 sparkSQL 会是比较好的选择.
\section{graphX :B}
\label{sec:org028f923}
用 graphX 做分布式图计算的需求.
\section{spark ml :C}
\label{sec:orgcda051b}
基于 spark ml 做一些模型(配合征信,金融,商服等需求)
\section{基于历史数据的快速交互式查询(可用 impala 代替) :D}
\label{sec:org652b9fd}
\section{Spark 适用场景:}
\label{sec:orgaacedc3}
Spark 是基于内存的迭代计算框架,适用于需要多次操作特定数据集的应用场合。需要反复操作的次数越多,所需读取的数据量越大,受益越大,数据量小但是计算密集度较大的场合,受益就相对较小。
数据量不是特别大,但是要求近实时统计分析需求
\section{Spark 不适用场景:}
\label{sec:orgcb90fce}
内存 hold 不住的场景,在内存不足的情况下,Spark 会下放到磁盘,会降低应有的性能
有高实时性要求的流式计算业务,例如实时性要求毫秒级
由于 RDD 设计上的只读特点,所以 Spark 对于待分析数据频繁变动的情景很难做(并不是不可以),比如题主例子里的搜索,假设你的数据集在频繁变化(不停增删改),而且又需要结果具有很强的一致性(不一致时间窗口很小),那么就不合适了。
流线长或文件流量非常大的数据集不适合。你会发现你的内存不够用,集群压力大时一旦一个 task 失败会导致他前面一条线所有的前置任务全部重跑,然后恶性循环会导致更多的 task 失败,整个 sparkapp 效率极低。就不如 MapReduce!
\end{document}
